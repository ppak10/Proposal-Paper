\chapter{Proposed Work and Timeline}
\label{chapter:proposal}

%---------------------------------------------------------------------

\section{
Proposed Work: Agentic Process Optimization for Selective Laser Sintering
}

Within the additive manufacturing process there are many operational and
environmental factors that can affect the build quality or build feasibility of
the final part \cite{huang_keyhole_2022, cacace_lack_2022,
tang_prediction_2017}. Mitigation of these effects is often achieved using
feedforward process control where model based approaches are applied to
anticipate problematic conditions that can lead to issues during the build
process \cite{bostan_accurate_2025, yeung_residual_2020, ogoke_thermal_2021}.
Feedback control addresses these issues during the build process utilizing
visual \cite{jadhav_llm-3d_2025}, thermal \cite{zhong_using_2021,
devesse_hardware---loop_2016}, or depth \cite{singh_toward_2023} information
obtained through various sensors to dynamically adjust parameters and toolpath
trajectories in an effort to resolve potential build defects. The use of
specialized LLMs, referred to as \textit{agents}, has enabled the intelligent
automation of tool orchestration and feedback control for complex tasks within
the space of additive manufacturing \cite{jadhav_llm-3d_2025, pak_agentic_2026}
as well as other fields such as catalyst discovery \cite{ock_adsorb-agent_2024}
and mechanical design \cite{jadhav_large_2024}. These advances enable automated
thought, action, and observation workflows within the research process,
utilizing perception and reasoning to make informed decisions to ensure
successful task execution \cite{ghafarollahi_atomagents_2024, pak_agentic_2026,
ock_adsorb-agent_2024, jadhav_llm-3d_2025}. In this proposed work an agentic
system enables the intelligent automation of parameter optimization and
\textit{in-situ} process monitoring during the selective laser sintering
process.

\begin{figure}[htbp]
    \centering
    \includegraphics[width=\textwidth]{chapters/5_proposal/agentic_sls_flow_portrait.png}
    \caption{
        Flow diagram depicts simplified user interaction with agentic system for
        parameter selection and process monitoring for the Inova Mk1.
        \textbf{(1)} The user prompt is provided as model input \textbf{(2)} and
  \textbf{(4)}       query for the vector database. Results from the vector database is
        appended to the model input and provided to a LLM. The LLM executes
        relevant \textbf{(3)} tool calls via MCP with their results added to the
        vector database. \textbf{(4)} Generated responses from the LLM are added
        to the vector database for use with future prompts.
    }
    \label{fig:agentic_sls_flow}
\end{figure}

\subsection{Experimental Platform}
\subsubsection{SLS4All Inova Mk1}
The selective laser sintering process will be performed using the Inova Mk1
(Fig. \ref{fig:inova_mk1}) for a range of materials such as PA12, PA12 GF, and
other experimental powder compositions. The Inova Mk1 is an open source, low
cost, selective laser sintering machine developed by SLS4All founders Tomas
Starek and Pavel Dyntera \cite{starek_sls4all_2020}. This machine was purchased
as kit from SLS4All and assembled over the course of several months producing
successful prints (Fig. \ref{fig:inova_print_proteins}) using Formlab's PA12 GF
\cite{formlabs_nylon_2026}. The Inova Mk1 utilizes a 450 nm blue diode laser
capable of delivering 10 watts of power \cite{starek_sls4all_2020}. The machine
is capable of an effective build volume of 150 mm x 150 mm x 185 mm and utilizes
an array of 4 halogen lamps for surface heating control and a 5 heating elements
for build chamber temperature control \cite{starek_sls4all_2020}. Surface
temperature monitoring and control is achieved with a \textit{ThermoCam
Waveshare MLX90640} \cite{noauthor_mlx90640_nodate} capable of producing a 32 x
24 pixel thermal image for temperatures ranging from 0 \textdegree C to 300
\textdegree C. The optical camera utilizes an Omnivision OV5647 sensor capable
of streaming a 1920 x 1080 pixel image at 30 frames per second. Average scan
speed is around 1,650 mm/s at 5 watts (250 \textmu m spot size) and 2,800 mm/s
at 10 watts (350 \textmu m spot size) \cite{starek_sls4all_2020}. The firmware
controlling the hardware components such as the galvometers, laser, stepper
motors, sensors, and heating elements runs off a combination of open-source
software programs including \texttt{Klipper}
\cite{kevinoconnor_klipper3dklipper_2026}, \texttt{SLS4All.Compact}
\cite{dyntera_sls4allsls4allcompact_2025}.

\begin{figure}[htbp]
    \centering
    
    \begin{subfigure}[b]{0.48\textwidth}
        \centering
        \includegraphics[width=\textwidth]{chapters/5_proposal/inova_mk1.JPEG}
        \caption{SLS4All Inova Mk1}
        \label{fig:inova_mk1}
    \end{subfigure}
    \hfill
    \begin{subfigure}[b]{0.48\textwidth}
        \centering
        \includegraphics[width=\textwidth]{chapters/5_proposal/inova_print_proteins.JPEG}
        \caption{Printed \textbeta2 Protein Model}
        \label{fig:inova_print_proteins}
    \end{subfigure}
    
    \caption{
    Inova Mk1 (a) assembled from kit prints complex geometries, such as the
    \textbeta2 protein molecule (b), only capable through the SLS process.
    }
    \label{fig:inova_mk1}
\end{figure}

\subsubsection{Platform Limitations}
The Inova Mk1 present several software and hardware limitations. Although the
software is primarily open-source, modules disclosing specific feature
implementations such as part slicing and tool path generation are considered
proprietary and only provided in a compiled state. In addition, there is little
documentation regarding the machine's exposed Application Programming Interface
(API) as the dashboard web interface is streamed from the host. With regard to
the installed sensors, relative to thermal and optical imaging equipment used in
other \textit{in-situ} process monitoring works
\cite{myers_high-resolution_2023, myers_two-color_2023, pak_thermopore_2024,
bostan_accurate_2025, ogoke_deep_2024}, the resolution and frame rate that these
sensors provide is comparatively coarse. This adds a potential constraint to the
quality of real-time information that can be utilized by the agentic system when
testing process parameters and executing builds. The manufacturer also suggest
material restrictions to primarily polymer based powders with a melting
temperature of 200 \textdegree C \cite{starek_sls4all_2020}.

\subsection{Agentic System}
The agentic system (Figure \ref{fig:agentic_sls_flow}) will allow for the
intelligent automation of parameter selection and process monitoring through the
guidance of a locally deployed large language model (potentially GPT-OSS 20b
\cite{openai_gpt-oss-120b_2025}). Integration with a vector database (i.e.
Pinecone) enables the use of dynamic memory such that previous system outputs
and tool call results can be recalled for future prompts, providing additional
context for the large language model to use during reasoning. Lastly, the system
is able to interact with its environment through agentic tool calls following
the Model Context Protocol. As indicated in Figure \ref{fig:agentic_sls_flow},
these include tools for material selection, simulation, print profile
configuration, parameter testing, and build monitoring.

\subsubsection{Material Selection Tool}
% TODO: Add more specific implementations
The material selection tool allows for inference on specified materials and
their approximate properties. For example, if provided a mixture of various
powders and additives, an approximation of critical temperature values such as
melting, glass transition, or solidus would be calculated. This would be
achieved through a various ensemble of modeling tools ranging from a simple rule
of mixtures to coarse resolution simulations depending on the argument
complexity.

\subsubsection{Simulation Tool}
% TODO: Add more specific implementations
The primary purpose of the simulation tool is to approximate the local
dimensions of the sintered material after the application of heat. Various
methods such as Finite Element Analysis \cite{kolossov_3d_2004} and other
numerical modeling techniques \cite{kandis_simulation-based_1999,
williams_advances_1998} specific to the selective laser sintering process and
materials will be used to gather general feasibility insight on process
parameters before testing.

\subsubsection{Parameter Testing Tool}
The parameter testing tool is a feature specific to the Inova Mk1 such that when
experimenting with new powders, a 5 x 5 grid of labeled patches (Figure
\ref{fig:inova_parameter_testing}) can be printed on the surface of the print
bed. This allows for the rapid testing of print parameters through patch
specific configurations such as outline and fill energy densities (Figure
\ref{fig:inova_parameter_testing_config}). In addition to executing patch
prints, the tool will utilize the available sensors to extract quality
information regarding each printed patch to determine which patch configuration
to test next.

\begin{figure}[htbp]
    \centering
    
    \begin{subfigure}[b]{0.56\textwidth}
        \centering
        \includegraphics[width=\textwidth]{chapters/5_proposal/inova_parameter_testing.JPEG}
        \caption{Printed Parameter Testing Patches}
        \label{fig:inova_parameter_testing}
    \end{subfigure}
    \hfill
    \begin{subfigure}[b]{0.4025\textwidth}
        \centering
        \includegraphics[width=\textwidth]{chapters/5_proposal/inova_parameter_testing_config.png}
        \caption{Patch Configuration}
        \label{fig:inova_parameter_testing_config}
    \end{subfigure}
    
    \caption{
        5 x 5 grid of patches (a) printed for testing PA12 GF powder, each with
        their own parameter configuration (b).
    }
    \label{fig:inova_mk1}
\end{figure}

\subsubsection{Print Profile Tool}
The print profile tool sets the various build parameters for a specific material
within the Inova Mk1 for future use. Notable parameters include surface and
chamber temperatures, energy densities for the fill and contours, desired layer
height, and recoater speed. This tool is expected to be executed once successful
use of the parameter testing tool is completed.

\subsubsection{Build Monitoring Tool}
The build monitoring tool will initialize the process for printing a specified
part along with monitoring the process for potential anomalies. This tool will
utilize the available sensors to observe each layer and terminate the build if
failure is detected.

\subsection{Evaluation}
The agentic system will be evaluated on a range of polymers and polymer based
composites with the mechanical properties tested to the ASTM 638-22 standard
\cite{d20_committee_test_nodate}. The setup will first determine the necessary
build parameters needed to print a specific material and validate the print
profile using the build monitoring tool. The evaluated prints will consist of
standard additive manufacturing benchmarking samples along with tensile
specimens.

%---------------------------------------------------------------------

\section{
Proposed Work: Continual Learning enabled Additive Manufacturing Large Language
Model
}

In this work a continual learning pipeline will be developed to further
integrate the collected results from agentic system into the large language
model. Continual learning is the ability for a system to incrementally update
and exploit available knowledge acquired throughout its lifetime
\cite{wang_comprehensive_2024, wu_continual_2024, scialom_fine-tuned_2022}. This
can be achieved through fine-tuning where the goal is to retain the existing
capabilities and avoid catastrophic forgetting while retraining with new updated
data \cite{scialom_fine-tuned_2022, jang_towards_2021, wang_comprehensive_2024}.
This catastrophic forgetting issue can be mitigated through augmenting the
original training data with the newly acquired data in a process called
Continual Learning via Rehearsal (CLR) \cite{scialom_fine-tuned_2022,
shin_continual_2017}

With this, dataset curation is of upmost importance to ensure fidelity in the
continual learning process. Background information regarding the expected
outcomes of this specific selective laser sintering process will be curated from
research articles and other technical specific to the polymer based SLS. Process
parameter related outcomes such as the dimensions of the sintered volume will be
sourced from simulations along with experimental literature finding if found.
Lastly, successful iterations of the previously mentioned agentic system will be
included as well in order to reinforce successful policy trajectories.
Integration of this continual learning pipeline is expected to enable enhanced
reasoning capabilities and this will be evaluated on the ratio of successful to
incorrect policy trajectories of the agentic system with and without continual
learning.

%---------------------------------------------------------------------

\section{Timeline}

My proposed milestones and timeline for completing the thesis are shown in Table~\ref{tb:milestone_timeline}.

\begin{table}[htb!]
  \centering
  \scriptsize
  \resizebox{\textwidth}{!}{\begin{tabular}{llllllll}
    \toprule
    Milestone & Timeline \\
    \midrule
    Data collection and System Development & February - April 2026 \\
    Experimental Testing and Evaluation & April - June 2026 \\
    Thesis Writing & June - July 2026 \\
    \midrule
    Thesis Defense & July 2026 \\
    \bottomrule
  \end{tabular}}
  \caption[Milestones and proposed timeline for completing the thesis]{Milestones and proposed timeline for completing the thesis.}
  \label{tb:milestone_timeline}
\end{table}
