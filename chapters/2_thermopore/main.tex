\newcommand{\bftab}{\fontseries{bs}\selectfont}

\chapter{ThermoPore: Predicting Part Porosity Based on Thermal Images Using Deep Learning}
\label{chapter:thermopore}

\section{Introduction}
Additive manufacturing (AM) presents a competitive alternative to the
conventional approaches in manufacturing with the advantages of efficient
material utilization, design consolidation, and fast iteration \cite{
rahman_review_2023, beaman_additive_2020, akbari_meltpoolnet_2022,
hemmasian_surrogate_2023}. However, a significant area of improvement lies
within defect prevention as printed parts present their own set of challenges in
porosity, distortion, and cracking\cite{rahman_review_2023}. These issues are
often uncovered through \textit{ex-situ} non-destructive testing methods and can
sometimes be addressed through lengthy post-processing means such as hot
isostatic pressing (HIPing) before they are certified
\cite{ordas_fabrication_2015, dolimont_effect_2016}. With \textit{in-situ}
process monitoring, a digital twin of the fabrication process can be created and
segments of the certification process can be conducted in
parallel\cite{knapp_building_2017, bartsch_digital_2021, gaikwad_toward_2020}.

Laser powder bed fusion (LPBF), relies primarily on established process maps
\cite{clymer_powervelocity_2017, agrawal_predictive_2022, grasso_-situ_2021} to
determine the optimal machine settings that minimize defects within the finished
part. Most commonly, these process maps explore the power and velocity space to
determine a combination of two that would result in a sufficiently dense part.
Informed control over these process parameters and others such as hatch spacing
\cite{xia_influence_2016}, layer height \cite{snyder_build_2015}, and raster
pattern \cite{mertens_optimization_2014}, can greatly affect the part's
porosity, microstructure \cite{gockel_understanding_2013}, and surface finish
\cite{snyder_build_2015}. However, even within build conditions with nominal
process parameters, defects such as porosity remain an issue
\cite{slotwinski_porosity_2014}.

\textit{In-situ} process monitoring offers a means to resolve this issue as
information obtained from the build process can assist in resolving many of the
technical challenges encountered during part fabrication
\cite{slotwinski_porosity_2014, tian_roadmap_2022}.
Many of these defects and their precursors such as part distortions
\cite{biegler_-situ_2018}, surface roughness \cite{hofman_situ_2022}, or keyhole
formation \cite{ren_machine_2023, tempelman_detection_2022} exhibit signals
which with the appropriate sensors can be detected before \textit{ex-situ}
sample analysis. These indicators can be applied alongside the build process to
analytical and machine learning models in order to obtain the necessary feedback
to adjust process parameters for the build. This feedback loop would be
optimized to reduce the number of part defects through both preemptive and
responsive measures. \cite{mccann_-situ_2021, feng_predicting_2022}. In
addition, reconstructing the porosity map can significantly accelerate the part
certification process as the knowledge of the porosity map can expedite
qualification through observations of statistics alone\cite{seifi_overview_2016,
chen_review_2022, knapp_building_2017, dordlofva_design_2020,
sola_microstructural_2019}. 

Thermal imaging demonstrates effectiveness as an \textit{in-situ} process
monitoring technique as evidenced by previous studies which have explored
comparing melt pool images to computational fluid dynamics simulations
\cite{myers_high-resolution_2023}, mathematical equations
\cite{kayacan_investigation_2020}, and 3D surface maps\cite{haley_-situ_2021}.
Further exploration of this technique has shown effectiveness in applications
such as defect detection and correction within the build process either
indicating likely porosity given a thermal image of a melt pool
\cite{mitchell_linking_2020} or material extrusion correction in large scale
additive manufacturing\cite{borish_real-time_2020, borish_-situ_2019}.

Analytical solutions such as Rosenthal's equation
\cite{rosenthal_mathematical_1941} provide a foundational method to determine
nominal process parameter regions within laser power and scanning velocity
space. This equation can be adapted to provide depth and width estimates of the
melt pool given specific process parameters such as preheat temperature, power,
and velocity which can be applied to the selection of nominal parameters for
hatch spacing and layer height. However, this method poses limitations as
solutions provided by Rosenthal's equation are only suitable for melt pools
within the conduction regime \cite{rosenthal_mathematical_1941,
hekmatjou_comparative_2020, imani_shahabad_extended_2021}. This leaves areas out
that are not captured through analytical models such as melt pool behavior in
the keyhole mode and process conditions such as scan strategies.

Much attention has been directed towards machine learning to fill this gap
between the projection of these analytical models and their applied results some
of which include process parameter optimizing
\cite{baturynska_optimization_2018, ogoke_inexpensive_2023} and fatigue life prediction
\cite{zhan_machine_2021}. In this paper we explore the application of machine
learning towards the quantification and spatial localization of pores within a
sample given the \textit{in-situ} monitoring data of thermal images. These
predictions can then be utilized to create a digital twin of the built sample
and perform qualification and certification tasks in parallel to the sample
fabrication\cite{gaikwad_toward_2020,bartsch_digital_2021,knapp_building_2017}.

For the task of pore quantification a three dimensional Convolutional Neural
Network (CNN) was utilized to extract features within a provided sequence of
thermal images and provide a singular scalar prediction of the expected number
of pores.  Models such as \textit{ImageNet}\cite{krizhevsky_imagenet_2017} have
shown the effectiveness of 2D CNNs with image classification tasks and other
models such as \textit{C3D}\cite{tran_learning_2015} have applied this technique
to extended over a sequence of images. Training a 3D CNN model with the
objective of pore quantification enables the identification of pore counts
within a build layer prior to any \textit{ex-situ} sample analysis.

The task of pore localization utilizes a Video Vision Transformer
(ViViT)\cite{arnab_vivit_2021} which is suited to capture the spatial and
temporal features within the sequence of thermal images through subdividing the
input into patches. The original implementation of the ViViT model is structured
to provide a classification output\cite{arnab_vivit_2021}, however for the
purposes of pore localization the classification head is replaced with a dense
prediction head which retains the spatial information of the input sequence. Our
network implementation utilizes a dense output which directly correlates the
spatial and temporal information into a 2D pore localization prediction. This
network builds off the work by Ranftl et al.\cite{ranftl_vision_2021} where
fusion blocks and convolutional layers are added to a vision transformer to
provide depth predictions of a given image.

Application of these aforementioned machine learning techniques alongside the
build process opens up the possibility to acquire expedited \textit{ex-situ}
sample insights and fabricate parts within a closed feedback loop. This has
the possibility to reduce labor and materials costs as parts fabricated
through laser powder bed fusion rely on \textit{ex-situ} post-build inspection
and testing to qualify parts and identify potential defects
\cite{damon_process_2018, feng_additive_2022, chen_review_2022,
donegan_multimodal_2021}. This is often a tedious process as
cross-sectional imaging or x-ray computed micro-tomography (CT) is required to
analyze these parts for defects such as keyholing or lack of fusion porosity.
\cite{ogoke_inexpensive_2023,myers_high-resolution_2023}. However, if
analogous information is obtained earlier during the build process through the
creation of a digital twin, problematic builds can be terminated earlier or
dynamic adjustments can be applied once the presence of defects is detected to
reduce material waste and costs.

Previous work towards establishing a correlation between the \textit{in-situ}
and \textit{ex-situ} dataset has been conducted with sensors such as thermal
monitoring, acoustic recording, or photodiode readings\cite{feng_additive_2022, 
rodriguez_approximation_2015, li_-situ_2024, coeck_prediction_2019}. With the
recent work by Li \textit{et al.}, the usage of acoustic \textit{in-situ}
monitoring was applied to recognize five laser powder bed fusion defects to an
accuracy of 99.12\%\cite{li_-situ_2024}, highlighting the effectiveness of
\textit{in-situ} process monitoring. Work by Coeck \textit{et
al.}\cite{coeck_prediction_2019} explores the effectiveness utilizing photodiode
sensors to determine lack of fusion porosity establishing a correlation directly
between \textit{in-situ} process monitoring to \textit{ex-situ} porosity
obtained with computed tomography. Our work explores an emerging approach to
correlate \textit{in-situ} pyrometry data to \textit{ex-situ} porosity through
the usage of deep learning approaches such as convolutional neural networks or
transformers.

For this purpose, we have constructed a digital twin framework called \textit{ThermoPore} for extrapolating
defect critical porosity information from a sequence of \textit{in-situ} thermal
images. This framework extends existing additive
manufacturing digital twin work, a data-based approach to develop
\textit{product twins} utilizing melt pool dynamic simulations and
\textit{in-situ} process monitoring techniques\cite{knapp_building_2017,
phua_digital_2022, zhang_digital_2020}. This work focuses on the utilization
\textit{in-situ} pyrometry data to construct characteristics for a
\textit{product twin} such as pore count and pore localization. As outlined in
Fig. \ref{fig:main}, the general architecture of this framework consists of two
separate deep learning models (Fig. \ref{fig:main}d and \ref{fig:main}e) which
extract embeddings from a sequence of \textit{in-situ} thermal images (Fig.
\ref{fig:main}c). These embeddings correspond to the quantitative and localized
information of pores obtained from the segmented computed tomography data (Fig.
\ref{fig:main}b). The predictions from these models indicate the degree of
porosity that can be anticipated from a given sequence of thermal images. By
leveraging the capabilities of Convolutional Neural Networks and Video Vision
Transformers, \textit{ThermoPore} enables efficient evaluation of laser powder
bed fusion printed parts.

\begin{figure}[bt]
  \centering
  \includegraphics[width=\textwidth]{chapters/2_thermopore/figure_1.pdf}
  \caption{
    A sequence of 200 thermal pyrometry images (\ref{fig:main}c) providing
    absolute temperature values of the build plate taken \textit{in-situ}
    (\ref{fig:main}a) are provided as input data for models for pore
    quantification (\ref{fig:main}d) and pore localization (\ref{fig:main}e).
    These two separate models utilize a CNN and ViViT with dense output heads to
    produce a scalar number of pores and 2D mapping of expected porosity regions
    respectively. Metrics derived from \textit{ex-situ} CT data for the
    corresponding build layer are used as ground truth values for each model
    (\ref{fig:main}b).
  }
  \label{fig:main}
\end{figure}

\section{Methodology}

\subsection{\textit{Spacing} and \textit{Velocity} Samples}
\subsubsection{Sample Fabrication and Data Acquisition}
This paper analyzes two samples, one with variable hatch spacing
(\textit{Spacing}) and the other with variable scan velocity
(\textit{Velocity}). Both of these samples were manufactured using LPBF
equipment (ProX DMP 200 from 3D Systems) with AISI 316L stainless steel powder
and a constant laser power of 103 W\cite{arnhart_-situ_2022,
mitchell_linking_2020}.  These samples were designed with a staircase structure
(Fig.  \ref{img:sample_design}) with each sample comprised of 10 separate steps
and each step consisting of a 16 build layers with a 30 {\textmu}m layer height.
Within each of these steps a different combination of process parameters were
implemented with changes to either hatch spacing (Fig.
\ref{tab:spacing_process_parameters}) or scanning velocity (Fig.
\ref{tab:velocity_process_parameters}). The expected dimensions of each sample
are 4.80 mm \texttimes\;2.80 mm \texttimes\;1.00 mm in height, length, and width
respectively\cite{arnhart_-situ_2022}. Each step consisted of dimensions 0.48 mm
in height and ranged from 1.00 mm to 2.80 mm in length, increasing in length by
0.20 mm from the top of the sample to the bottom. The \textit{Spacing} sample
with varying hatch spacing was built with a constant 1.4 m/s scan velocity and
the \textit{Velocity} sample with varying scan velocity was built with a
constant 50 {\textmu}m hatch spacing \cite{arnhart_-situ_2022}.A \textit{normal}
rastering pattern consisting of line scans parallel to the build axes,
orthogonal to the previous layer was utilized as the scan strategy for both
samples.

\begin{figure}[bt]
  \centering
  \begin{subfigure}{0.20\textheight}
      \centering
      \includegraphics[width=\textwidth]{chapters/2_thermopore/sample_design.jpeg}
      \caption{Sample Design}
      \label{img:sample_design}
  \end{subfigure}%
  \begin{subfigure}{0.33\textwidth}
      \centering
      \begin{tabular}{|c|c|}
          \hline
          \small\textbf{Step} & \small\textbf{Hatch Spacing} \\
          \hline
          J & 25 {\textmu}m \\
          \hline
          I & 30 {\textmu}m \\
          \hline
          H & 35 {\textmu}m \\
          \hline
          G & 40 {\textmu}m \\
          \hline
          F & 45 {\textmu}m \\
          \hline
          E & 75 {\textmu}m \\
          \hline
          D & 70 {\textmu}m \\
          \hline
          C & 65 {\textmu}m \\
          \hline
          B & 60 {\textmu}m \\
          \hline
          A & 55 {\textmu}m \\
          \hline
      \end{tabular}
      \caption{\textit{Spacing} Parameters}
      \label{tab:spacing_process_parameters}
  \end{subfigure}%
  \begin{subfigure}{0.33\textwidth}
      \centering
      \begin{tabular}{|c|c|}
          \hline
          \small\textbf{Step} & \small\textbf{Scan Velocity} \\
          \hline
          J & 1.05 m/s \\
          \hline
          I & 1.12 m/s \\
          \hline
          H & 1.19 m/s \\
          \hline
          G & 1.26 m/s \\
          \hline
          F & 1.33 m/s \\
          \hline
          E & 1.75 m/s \\
          \hline
          D & 1.68 m/s \\
          \hline
          C & 1.61 m/s \\
          \hline
          B & 1.54 m/s \\
          \hline
          A & 1.47 m/s \\
          \hline
      \end{tabular}
      \caption{\textit{Velocity} Parameters}
      \label{tab:velocity_process_parameters}
  \end{subfigure}
  \caption{
    Each sample (\ref{img:sample_design}) contains 10 different
    process parameter combinations  with the \textit{Spacing} sample
    (\ref{tab:spacing_process_parameters}) exhibiting varying hatch spacing
    and the \textit{Velocity} sample (\ref{tab:velocity_process_parameters})
    exhibiting varying scan velocity.
  }
  \label{fig:sample_fabrication}
\end{figure}

\subsubsection{In-situ Pyrometry}
Absolute temperature estimations were calculated from thermal radiation captured
by a Stratonics two-color pyrometer receptive to light emitted at 750 nm and 900
nm, calibrated with NIST-traceable tungsten lamp\cite{mitchell_linking_2020}.
Images were captured with a frame rate within 6 - 7 kHz and a 90 {\textmu}s
exposure.\cite{mitchell_linking_2020} Temperature estimations without in depth knowledge of emissivity parameters were performed with a grey-body assumption.\cite{mitchell_linking_2020}.
Synchronization between the LPBF equipment and pyrometer were achieved via
Transistor-Transistor Logic (TTL) triggering producing 1000 frames of 65 px
{\texttimes} 80 px images within each build layer\cite{mitchell_linking_2020}.
This translated to a 1365 {\textmu}m {\texttimes} 1680 {\textmu}m resolution
with approximately 21 {\textmu}m per px. This presented a total of 159,000
images taken for each sample and with initial screening applied to filter out
"empty" images, reducing the total number of frames down to 20,469 and 20,187
for \textit{Spacing} and \textit{Velocity} samples respectively.
Within the \textit{Spacing} sample, the melt pool
temperatures observed a consistent spread between 1400 $^{\circ}$C to 1650
$^{\circ}$C through all build layers, which aligns with the expected behavior of
parts fabricated with constant power and velocity.

Within the range of potential melt pool temperatures, lack of fusion porosity is
expected toward the lower temperature bound due to insufficient melting and
keyhole porosity at the upper temperature bound. Keyhole porosity results from
the collapse of the rear keyhole, driven by recoil pressure, which peaks at the
vaporization temperature of the material but is also influenced by ambient
pressure \cite{wang_mechanism_2022}. Studies have shown that lower ambient
pressures reduce porosity by lowering the vaporization temperature, thus
affecting recoil pressures \cite{zhang_microstructure_2013, zhou_study_2018,
jiang_mitigation_2020}. At the lower bound, lack of fusion porosity occurs when
melt pool temperatures are below the material's melting point, influenced by
factors like melt pool dimensions, layer height, and hatch
spacing\cite{tang_prediction_2017}. For 316L Stainless Steel, with a liquidus
temperature of 1437.11 $^{\circ}$C (1710.26 K), solidus temperature of 1410.53
$^{\circ}$C (1683.68 K), and vaporization temperature of 2860.85 $^{\circ}$C
(3134 K)\cite{miyata_inverse_2021}, these values establish the temperature
boundaries for avoiding porosity. However, due to various factors like ambient
pressure affecting the vaporization temperature, machine learning models are
used to analyze temperature sequences and predict porosity in computed
tomography data.

\subsubsection{Ex-situ Micro-computed Tomography (CT)}
Micro-computed tomography analysis was performed using a Zeiss Xradia 520 Versa
at the maximum output power of 10 W and tube voltage of 140 kV with the sample
positioned 11.1 mm from the source \cite{arnhart_-situ_2022}. Scans were taken
at a cubic voxel size of 3.63 \textmu m and with a build layer height of 30
\textmu m, this equated to approximately 8.26 voxels per build layer. The
obtained scans for the \textit{Spacing} and \textit{Velocity} samples were
bounded in the Z \texttimes\;Y \texttimes\;X directions by 5.05 mm
\texttimes\;3.00 mm \texttimes\;1.35 mm and 5.05 mm \texttimes\;3.17 mm
\texttimes\;1.44 mm respectively (Fig. \ref{fig:sample_alignment}a)
\cite{arnhart_-situ_2022}. The extracted 3D representation for both
\textit{Spacing} and \textit{Velocity} samples extended 1410 voxels
\texttimes\;900 voxels \texttimes\;430 voxels along the Z, Y, and X axes.

With the voxelized dataset obtained from both the \textit{Spacing} and
\textit{Velocity} samples, further actions were performed to extract porosity
attributes. This included each pore's unique identifier, equivalent
diameter, centroid, number of voxels, and minor and major axes. Of these traits,
the pore's unique identifier was primarily used to determine the boundaries of
each individual pore which was further processed to labels used for the Pore
Quantification and Pore Localization tasks.

\begin{figure}[bt]
  \centering
  \includegraphics[width=\textwidth]{chapters/2_thermopore/figure_3.pdf}
  \caption{
    Visualization of segmented CT porosity within overall sample for
    \textit{Spacing} and \textit{Velocity} samples
    (\ref{fig:sample_alignment}a). Figure \ref{fig:sample_alignment}b
    superimposes the pyrometry image directly over the corresponding CT data
    with alignment offsets applied along the X and Y axes, using the top left
    corner of both as the origin. Alignment for a build layer is visually
    validated (\ref{fig:sample_alignment}c) as only a subset of the built sample is
    visible through the lens of the pyrometer (\ref{fig:sample_alignment}d).
  }
  \label{fig:sample_alignment}
\end{figure}

\subsection{Pyrometry and Micro-computed Tomography Datasets}
\subsubsection{Pyrometry and Micro-computed Tomography Alignment}
\subsubsection{Data Alignment}
To ensure accurate correlation between pyrometry input and corresponding
micro-computed tomography pore labels, effective alignment of the two datasets
is essential. In the pyrometry dataset, the capture area (Fig.
\ref{fig:sample_alignment}c) needs to be considered as it is limited to 80 px
\texttimes\;65 px (1680 \textmu m \texttimes\;1365 \textmu m) area of the
sample, leading to raster patterns of lower build layers (steps A - F) to extend
further than the camera's viewport \cite{mitchell_linking_2020}. As mentioned,
initial screening filtered out many of these empty frames by applying a minimum
threshold on each frame's peak value from long wavelength
data\cite{mitchell_linking_2020}. This method removed the frames where the melt
pool was out of view, however still included some frames where spatter likely
occurred.

\textit{In-situ} thermal images and slices of \textit{ex-situ} CT data shared
the same origin at the top left (Fig \ref{fig:sample_alignment} b). Small
offsets were then applied to align the X and Y directions of CT to the thermal
image. The provided Z direction offsets were used as starting points to align
the CT data to the corresponding build layer. The Z alignment for both samples
were visually verified through manual alignment of the scan path of thermal
images and the a top down view of the corresponding CT layer. Step G (Fig.
\ref{img:sample_design}) within both samples was the first section of the sample
where the entire scan path is in complete view of the thermal camera and was
used as a reference point to align the CT data. Both samples (\textit{Spacing}
and \textit{Velocity}) were offset by a total 5 build layers (\textasciitilde 9
voxels per build layer) in total.

\subsubsection{Pore Thresholding}
A CT resolution of 3.63 \textmu m per voxel allowed for the capture of distinct
shapes and contours associated with porosity, however this fine resolution also
resulted in recording scattered distributions of small voids. Accurately
predicting these small voids is a difficult task for a model as these defects
could be the result of gas porosity \cite{iebba_influence_2017} or rogue flaws
\cite{reutzel_application_2023} and may have precursors not visible to thermal
imaging. To achieve greater correlation between the pyrometry data and CT data,
our attention focused on larger diameter pores that can be attributed to factors
such keyhole porosity or lack of fusion porosity. In keyholing, pores generated
by the vapor column during builds resulted in an average diameter of 47 \textmu
m \cite{shrestha_study_2019} and lack of fusion pores with diameters dependent
on the height and width of melt pool and corresponding build layer
\cite{cacace_lack_2022}.

\begin{figure}[bt]
  \centering
  \includegraphics[width=\textwidth]{chapters/2_thermopore/figure_4.pdf}
  \caption{
    The \textit{Spacing} sample on average consist of larger pores with a
    standard deviation of 20.5 \textmu m compared to the \textit{Velocity}
    sample's standard deviation of 16.0 \textmu m
    (\ref{fig:porosity_threshold}a). The tighter distribution of
    \textit{Velocity} pores are visible in the thresholded distribution of
    equivalent pore sizes (\ref{fig:porosity_threshold}b) within the slices
    of the CT data. The segmentation of pores within the sample
    (\ref{fig:porosity_threshold}c) relied on a minimum of 100 voxels (11.4
    \textmu m equivalent spherical diameter)\cite{mitchell_linking_2020} in
    order to register the contiguous cluster of voids as porosity
    (\ref{fig:porosity_threshold}d) and an increase in the minimum size
    further removed smaller pores (\ref{fig:porosity_threshold}e).
  }
  \label{fig:porosity_threshold}
\end{figure}

The mean Equivalent Spherical Diameter (ESD) for each sample was compiled in
order to obtain minimum threshold values 1 standard deviation above the mean. In
the \textit{Spacing} sample, pores exhibited an average diameter of 32.59
\textmu m and a standard deviation of 20.60 \textmu m, resulting in a minimum
(\textit{$\mu + \sigma$}) ESD threshold of 53.19 \textmu m.  In the
\textit{Velocity} sample, pores had an average diameter of 30.78 \textmu m and a
smaller standard deviation of 16.01 \textmu m resulting in a minimum
(\textit{$\mu + \sigma$}) ESD thresholds of 46.79 (Fig.
\ref{fig:porosity_threshold}b).

\section{Porosity Reconstruction}
The quantity of pores and their position within the build layer was
reconstructed with machine learning inferencing upon a sequence of thermal
images.  With the set of \textit{in-situ} and \textit{ex-situ} data, the tasks
of predicting the number of pores and the approximate location of these pores
within the various build layers were achieved using a CNN model and a ViViT
model with dense prediction heads respectively.

The number of pores corresponding to a sequence of thermal images were predicted
using a CNN model. In this task both the application of rotational transforms on
the input sequence and the volumetric depth utilized for pore count were treated
as variables. In some cases pores, such as those resulting from keyhole porosity
\cite{wang_mechanism_2022}, can form below the build layer. Also within the CT
data, there exist pores which extend beyond one build layer into multiple
build layers. In order to account for this, labels were derived from counting
the set of unique pores within a specified volumetric depth corresponding to 1,
2 or 3 build layers below the thermal image by referencing each voxel's pore id.
The quality of these predictions is measured using Root Mean Square Error (RMSE)
and $\text{R}^2$ score.

The localization of pores was predicted using the same sequence of thermal
images and utilized a ViViT model with a dense prediction head to indicate
sections expected to be porous. The labels for this task were obtained by
downsampling the CT data for the build layer equally along all axes in order to
provide a coarse porosity estimates for the model to train and predict. In
addition, the effect of applying rotational transforms on the input sequence and
minimum pore ESD thresholds for label compilation were investigated. The quality
of the predictions was measured using an Intersection over Union (IoU) score,
considering the overlap of the area of predicted porosity over that of the
label.

\subsection{Porosity Count}
This task investigates the extent in which sequences of thermal images can
quantify the number of pores that exist within the sample build layers. Without
the application of a $\mu + \sigma$ equivalent diameter threshold, the
\textit{Spacing} sample consists of 2069 pores and the \textit{Velocity} sample
consists of 1811 pores. The input for this model consists of a sequence of 200
64 \texttimes\;64 pixel thermal images of the build layer. The labels were
obtained from corresponding build layer region of the CT sample data, where the
number of unique pore identifiers were counted. Each build layer consisted
of 9 voxels in depth and this volume was extended to a depth of 18 and 27 voxels
to obtain pore counts extending down 2 and 3 build layers.

In this task a CNN model composed of 4 convolutional layers and 2 fully
connected layers reduce the input set of 200 64 \texttimes\;64 pixel images into
a scalar value of the number of unique pores within the build layer (Fig.
\ref{fig:cnn}). A 3 \texttimes\;3 pixel kernel is convolved on top each image
with a stride of 2 and a padding of 1. Within each layer the number of channels
is reduced by a factor of 2 and batch normalization and ReLU non-linearity
activation function are applied. The output of the CNN layers are reshaped into
a 2 dimensional tensor before they are passed to the two fully connected layers
which output a single scalar value which quantifies the number of pores within
the build layer.

\begin{figure}[bt]
  \centering
  \includegraphics[width=\textwidth]{chapters/2_thermopore/cnn.png}
  \caption{
    A standard convolutional neural network with a kernel size of 3 pixels and a
    stride of 2 filters the features within an image down over 4 layers to a
    singular scalar value indicating the expected number of pores from the
    sequence of thermal images.
  }
  \label{fig:cnn}
\end{figure}

A CNN operates by leveraging convolutional layers to extract features from input
images hierarchically. In the initial layers, low level features such as edges
and gradients are detected through convolutional operations where the kernel
moves across the input image, computing dot products and producing feature maps
\cite{krizhevsky_imagenet_2017, tran_learning_2015, long_fully_2015,
jadhav_stressd_2023}. Activation functions such as ReLU apply non-linear
transforms and allow for the capture of complex patterns. The following layers
then build upon these low level feature maps and repeat this task of
transforming the raw pixel values into the outputs for a specific task
\cite{krizhevsky_imagenet_2017}. These tasks are determined by the final layers
of the network which utilize fully connected layers to shape the output to
perform classification, regression, or reconstruction.

Due to dataset size constraints, a 75 / 25 train test split of the data was used
for model training. The train test split occurred within the 16 build layers of
each of the 10 sample steps of either the \textit{Spacing} or \textit{Velocity}
sample. This split within sample steps was implemented to provide an equal
distribution of processing parameters between the train and test sets for the
model. This provided an input label set of 120 training pairs and 39 testing
pairs for the either of the samples. Our dataset implementation allowed for the
model to train on either the \textit{Spacing}, \textit{Velocity}, or on a
combination of both datasets with the \textit{All} dataset.  For this regression
task, a mean squared error was utilized as the loss function and the predicted
value from the model is rounded to the nearest integer. Each of the models were
trained from 500 epochs with a learning rate of 0.0001 using the ADAM optimizer.

In addition, data augmentations of the input sequence in the form of rotational
transforms for the entire video sequence were applied. Data augmentation
provides a means to artificially expand an existing set of data in order to
improve the generalization ability and robustness of the
model\cite{krizhevsky_imagenet_2017}. Typical transformations change the input
image applying one or a combination of rotations, translations, flips, scaling,
and cropping. In this application we focus only on applying rotational
transforms ranging between $0^{\circ}$ to $180^{\circ}$ and avoid alterations to
the contrast or brightness that would affect the raw temperature value.

\subsection{Porosity Localization}
The localization task identifies areas within the build layer where pores are
likely to form through analyzing the build layer's sequence of 200 64
\texttimes\;64 pixel thermal images. A video vision transformer (ViViT) provides
an applicable architecture to thoroughly analyze the series of input frames to
extract positional features through use of spatial and temporal
attention.\cite{arnab_vivit_2021} In our model implementation a sequence of
thermal images for a specified build layer is provided to the model to map to
localized porosity labels obtained through alignment and extraction of the CT
data. Within the CT data a build layer is a size of 9 \texttimes\;423
\texttimes\;520 voxels. The areas that indicate porosity area then extracted and
downsampled by a factor of 24 to a coarser 2 dimensional 1 \texttimes\;16
\texttimes\;16 voxel shape to provide a general area in which porosity is
expected (Fig. \ref{fig:input_cropping_label_downsampling}).

For this task a video vision transformer model with a dense prediction output is
utilized to localize pores within the sample space. This model is composed of a
spatial transformer layer with 4 sub-layers and a temporal transformer layer
with 5 sub-layers both with 8 heads and a dimension of 256 (Fig.
\ref{fig:vivit}). Afterwards the class tokens resulting from each of the
transformer layers are removed and the output is passed into a feature fusion
block which performs residual convolution and provide a fine grain prediction. A
series of 4 convolutional layers and ReLU non-linearity layers are applied
before ultimately passing through a sigmoid activation function.  The sigmoid
activation function constrains the output between 0 and 1, providing a
probability distribution of the existence of porosity at this location of the
sample. To compare the outputs of this model to the ground truth, the outputs
are converted to binary representations to the existence of porosity, which
allows for the calculation of IoU performance metrics. 

\begin{figure}[bt]
  \centering
  \includegraphics[width=\textwidth]{chapters/2_thermopore/vivit_dense.png}
  \caption{
    The input sequence of thermal images are sliced into a set of patches
    (\ref{fig:vivit}a) where 4 and 5 sublayers of the respective spatial and
    temporal self attention are applied (\ref{fig:vivit}b). Each self attention
    block consisted of 8 attention heads (\ref{fig:vivit}c) of a dimension of
    256. A feature fusion block (\ref{fig:vivit}d) applies residual convolution
    and a dense prediction head produces a 2 dimensional output indicating
    regions of expected porosity.
  }
  \label{fig:vivit}
\end{figure}

Attention within a transformer is comprised of three learnable components: The
query vector $\{\textbf{q}_i\}^{N_q}_{i=1}$, key vector
$\{\textbf{k}_i\}^{N_k}_{i=1}$, and the value vector
$\{\textbf{v}_i\}^{N_v}_{i=1}$ given that $N_k = N_v$\cite{bahdanau_neural_2016,
vaswani_attention_2023, li_scalable_2024}. In the attention mechanism, the query
vector retrieves contextual information from the key vector and generates an
output based on the weighted sum of corresponding value vectors. Contextual
information is retrieved in the form of an attention score which is the scaled
dot product between the query vector and the key vector:
$\frac{\textbf{q}_i\cdot\textbf{k}_j^T}{\sqrt{d_k}}$\cite{vaswani_attention_2023}.
The attention score is then used in the calculation of weights ($\alpha_{ij}$)
which apply a Softmax over the individual contribution of each attention score
(Eq. \ref{eq:attention}) \cite{vaswani_attention_2023, bahdanau_neural_2016,
li_scalable_2024}.Each token's numerical encoding along with its relevance to
other tokens is calculated from the cross product between the value vector and
weights resulting in the attention mechanism:
$Softmax\left(\frac
{\textbf{Q} \cdot \textbf{K}^\textbf{T}}
{\sqrt{d_k}}
\right)\times V$
\cite{vaswani_attention_2023, bahdanau_neural_2016, li_scalable_2024}.

\begin{equation}
\alpha_{ij} = Softmax\left(\frac {\textbf{Q} \cdot \textbf{K}^\textbf{T}}
{\sqrt{d_k}} \right) = \frac
{e^{\frac{\textbf{q}_i\cdot\textbf{k}_j^T}{\sqrt{d_k}}}}
{\sum^N_{k=1}e^{\frac{\textbf{q}_i\cdot\textbf{k}_k^T}{\sqrt{d_k}}}}
\label{eq:attention}
\end{equation}

In the case of a Vision Transformer (ViT), an image is divided up into self
attention patches via patch embedding, passed through the transformer encoder,
and utilized through a classification head\cite{dosovitskiy_image_2021,
arnab_vivit_2021}.  Within patch embedding, each patch has a fixed pixel size in
height and width and its embedding is derived through flattening and linear
projection\cite{dosovitskiy_image_2021, arnab_vivit_2021}.  Class and position
are applied to the embedding before passed through the transformer encoder
composed blocks of layer norm, multi-head attention, and MLP layers after which
a classification head is attached.\cite{dosovitskiy_image_2021}

The dataset utilizes the same 75 / 25 train test split within the sample steps
for all the \textit{Spacing}, \textit{Velocity}, and \textit{All} variants. The
model was trained for 1000 epochs utilizing a binary cross entropy loss function
for the binary prediction of a voxel's porosity classification. An ADAM
optimizer along with a cosine decay learning rate scheduler with an initial 10
epoch warm up period from learning rates 0.00001 to 0.0001 is applied to help
with regularization and stabilization \cite{gotmare_closer_2018}. 

\section{Results and Discussion}

\subsection{Porosity Count}
The number of pores within a volume of the sample was predicted from a sequence
of thermal images using a CNN model trained on either the \textit{Spacing},
\textit{Velocity}, or \textit{All} dataset. Hyperparameters such as the
utilization of rotational transforms and the number of build layers involved in
the compilation of the pore count were investigated.

Table \ref{tab:pore_count_cnn_results} outlines the model's performance
according to its training dataset and indicates the data augmentation procedures
that were applied and the various depths used to calculate pore count. Models
with datasets spanning 1 build layer exhibited the lowest error and highest
$\text{R}^2$ with a minimum RMSE score of 7.84 from the \textit{Velocity} sample
and a maximum $\text{R}^2$ score of 0.57 from the \textit{Spacing} sample. The
RMSE aims to measure the average magnitude of errors such as the degree in which
the prediction deviates from the target. The $\text{R}^2$ score captures the
proportion of variance that can be attributed to explanatory variables. The
alignment of the prediction and the corresponding target for each model (Fig.\ref{fig:pore_count_results}) shows a general trend where there is a larger
degree of porosity within the build layers ranging from 60 to 100 and to a
lesser degree elsewhere. Notably, build layers greater than 100 display lower
amounts of porosity for all models regardless of dataset or data augmentation.

This is expected as both samples transition from non-nominal process parameters
to nominal process parameters towards the middle of the sample (Table
\ref{tab:spacing_process_parameters}, \ref*{tab:velocity_process_parameters})
with the upper portion of each sample fabricated with ideal process parameters.
For all models, areas where the sample was fabricated with nominal process
parameters show a greater degree of clustering as pores are sparse and few. In
earlier build layers, specifically those closer towards the middle of the sample
a greater spread of predictions is seen (Fig. \ref{fig:pore_count_results}).

The model trained with the \textit{Spacing} dataset without rotational
transforms (Fig. \ref{fig:pore_count_results} Bottom) achieved the highest
$\text{R}^2$ score of all models with a score of 0.57. This indicates that the
more than half of the variance within this model is able to be explained by the
input data. However, the model trained with the \textit{Velocity} sample and
rotational transforms achieved the lowest RMSE score of 7.84 of all models on
input data with significantly less variability as indicated with their lower
$\text{R}^2$ scores. The model trained on the \textit{All} dataset produced RMSE
and $\text{R}^2$ scores inbetween that of models trained on either
\textit{Spacing} or \textit{Velocity} datasets except in the situation for the
case where rotational transforms were applied where it yielded a $R^2$ score of
0.07. The additional hyperparameter of various build layer depths displayed
mostly worse RMSE and $\text{R}^2$ scores indicating that there exists a greater
correlation between the thermal images and the build layer directly underneath
it (Table \ref{tab:pore_count_cnn_results}). The exception to trend occurs
within the \textit{Velocity} dataset where $\text{R}^2$ score (0.36) observes a
visible increase for label pairs extending to 2 build layers. This is seen to a
lesser extent for porosity expanding to 3 build layers and aligns with the
findings by Wang \textit{et al.}\cite{wang_mechanism_2022} that keyhole porosity
can travel well below the immediate build layer.

The RMSE and $\text{R}^2$ values align with what is expected of the two
\textit{Spacing} and \textit{Velocity} datasets as the hatch spacing and scan
velocity are the two variables that change between sample steps respectively. In
the case of \textit{Spacing} sample, the hatch spacing produces a visible signal
in the form total rasters that is visible over the sequence of input frames.
However, the \textit{Velocity} sample uses a consistent number of rasters
traveling both vertically and horizontally across the build plate. The most
significant visual signal in the \textit{Velocity} sample is the distance the
melt pool travels inbetween frames.

Within existing literature, our pore count model performs comparably to a
similar study conducted by Coeck \textit{et al.}\cite{coeck_prediction_2019}
which detected the presence of porosity within 54 of the total 93 porosity
samples before the removal of false positive results. Given the limited size
and resolution of the training and testing dataset, the range of observed melt
pools is restricted. Although an experiment for creating a dataset varying in
power process parameters was conducted, challenges encountered during the data
collection process prevented its use within our datasets. A larger dataset
along with a higher resolution image would offer more examples and features for
the model to train with and improve RMSE and $R^2$ performance metrics. 

\begin{figure}[bt]
  \centering
  \includesvg[width=\textwidth]{chapters/2_thermopore/figure_7.svg}
  \caption{
    CNN pore count predictions plotted alongside target pore values for
    \textit{Spacing}, \textit{Velocity}, and \textit{All} (combination of both
    datasets) for a depth of 1 build layer. In all plots the pore counts for
    build layers 100 - 160 are clustered near the origin as the upper half of
    the sample is composed of nominal process conditions. Of all the various
    implementations, the CNN model trained on the \textit{Spacing} (Top Left)
    dataset without transforms shows the greatest level of alignment between the
    prediction and ground truth.
  }
  \label{fig:pore_count_results}
\end{figure}

\begin{table}
  \begin{tabular}{ll|ll|ll|ll}
    \hline
    \rowcolor{gray!30}
                        &                   & \multicolumn{2}{|l}{\textit{1 Build Layer}}   & \multicolumn{2}{|l}{\textit{2 Build Layers}}  & \multicolumn{2}{|l}{\textit{3 Build Layers}}  \\
    \rowcolor{gray!30}
    Dataset             & Data Augmentation & RMSE            & $\text{R}^2$                & RMSE            & $\text{R}^2$                & RMSE            & $\text{R}^2$                \\
    \hline                                                                                                                      
    \textit{Spacing}    &  Rotational       & 12.60           & 0.33                        & 19.04           & 0.18                        & 22.94           & 0.17                        \\
                        &  None             & 10.14           & \textbf{0.57}               & 18.46           & 0.23                        & 19.75           & \textbf{0.39}               \\
    \rowcolor{gray!10}                                                                                                          
    \textit{Velocity}   &  Rotational       & \textbf{7.84}   & 0.09                        & 11.41           & -0.04                       & 14.74           & 0.03                        \\
    \rowcolor{gray!10}                                                                                                          
                        &  None             & 8.08            & 0.03                        & \textbf{8.95}   & \textbf{0.36}               & \textbf{13.25}  & 0.22                        \\
    \textit{All}        &  Rotational       & 11.90           & 0.07                        & 13.89           & 0.34                        & 17.81           & 0.26                        \\
                        &  None             & 9.73            & 0.38                        & 14.57           & 0.27                        & 16.88           & 0.34                        \\
    \hline
  \end{tabular}
  \caption{
    $\text{R}^2$ and RMSE prediction performance metrics of pore counts for the
    pore quantification task utilizing CNN models trained on various datasets
    and build layer depths for 500 epochs.
  }
  \label{tab:pore_count_cnn_results}
\end{table}

\subsection{Porosity Localization}

During the training process, rotation transforms of the entire video sequence
were introduced as data augmentation methods and compared against models trained
without any rotational transformations. In addition to data augmentations, the
datasets were adjusted to allow for training on \textit{All Pores} and on pores
with ESD 1 standard deviation above the mean (\textit{$\mu + \sigma$ Pores}) in
an effort to investigate the model's performance on identifying the larger pores
within the dataset.

The position of the pores within the build layer was predicting using a sequence
of thermal images with the ViViT Dense model trained on either the
\textit{Spacing}, \textit{Velocity}, and \textit{All} dataset. The effect of
data augmentation and applying a threshold for minimum ESD for
pores were investigated with the training of this model as well. The
Intersection over Union (IoU) also known as the Jaccard Index was used to
quantify the performance of each model's prediction. The intersection over union
quantifies the prediction area overlap onto the target with the highest metric
of 1.0 occurring from an exact overlap of the two sets. For the IoU calculation
each set only includes the areas of porosity and in the case where both the
target and prediction exhibited no porosity, a score of 1.0 was given as the
prediction provided an exact match of the ground truth. 

\begin{figure}[bt]
  \centering
  \begin{subfigure}[b]{0.45\textwidth}
    \centering
    \includesvg[scale=0.445]{chapters/2_thermopore/figure_8a.svg}
    \caption{\textit{IoU Overlay with \textit{All Pores}}}
    \label{fig:iou}
  \end{subfigure}
  \begin{subfigure}[b]{0.45\textwidth}
    \centering
    \includesvg[scale=0.13]{chapters/2_thermopore/figure_8b.svg}
    \caption{\textit{IoU Overlay with \textit{$\mu + \sigma$ Pores}}}
    \label{fig:iou_threshold}
  \end{subfigure}
  \caption{
    Comparison of prediction and label values for the model trained on the
    \textit{All} samples dataset alongside the application of a minimum ESD pore
    threshold (Fig. \ref{fig:iou_threshold}). The examples correspond to build
    layer number 16 of the \textit{Velocity} sample and the areas colored in
    yellow represent the intersection that is considered for IoU calculations
    (ignoring the background). The ViViT model trained on pores without
    thresholding (Fig. \ref{fig:iou}) achieved an IoU score of 0.672 whereas the
    model trained with \textit{$\mu + \sigma$} thresholded pores (Fig.
    \ref{fig:iou_threshold}) produced a lower IoU score of 0.323.
  }
  \label{fig:ious}
\end{figure}

The effect of training on the various \textit{Spacing}, \textit{Velocity}, and
\textit{All} datasets were investigated for this task as well the impact of 
rotational transforms on the input sequence. In addition, the prediction
performance of the model upon applying a threshold to hide pores smaller than 1
standard deviation above the mean was applied. The prediction results of these
models measured for each input label pair within the test dataset with the
overall average and maximum IoU scores recorded on Table
\ref{tab:pore_localization_vivit_dense_results}. Within the datasets that
included \textit{All Pores}, models trained on the \textit{Spacing} datasets
performed the best on average IoU scores when trained without rotational
transforms. A maximum IoU score of 0.85 was achieved within the model trained on
the dataset for \textit{All} samples with rotational transforms applied. Without
the applied threshold for minimum pore ESD, the model trained on the
\textit{Velocity} produces the lowest IoU score of the three models (Fig.
\ref{fig:pore_localization_results}). One explanation for the higher IoU trend
seen in the \textit{Spacing} dataset compared to that of the \textit{Velocity}
dataset is that the \textit{Spacing} sample has a higher degree of porosity
compared to that of the \textit{Velocity} sample. This can be attributed to lack
of fusion pores within the \textit{Spacing} dataset which are often larger than
that of the keyhole pores in the \textit{Velocity} dataset. When
viewing the IoU trends in Figure \ref{fig:pore_localization_results}, this is
reflected where the lower layers of the sample achieve higher IoU scores than
that of the upper layers of the sample.

However, after a minimum pore ESD threshold is applied, the \textit{Velocity}
dataset model performs on par to that of the model trained with the
\textit{Spacing} dataset (Fig.
\ref{fig:pore_localization_results_thresholded}). In all of the cases there was
a greater trend in overlap within the lower layers of the sample (Fig.
\ref{fig:iou}) likely due to the higher porosity resulting from the non-nominal
process conditions used within those sample steps.  With a minimum pore
threshold of 1 standard deviation above the mean pore equivalent diameter
(\textit{$\mu + \sigma$ Pores}), all models were able to achieve maximum IoU
scores of 1 (Fig. \ref{fig:pore_localization_results_thresholded}). These
results were achieved towards the top of each sample where nominal process
conditions were used and correctly predicted the ground truth of no pores above
the threshold were present. (Fig.  \ref{fig:iou_threshold}) In some cases, areas
with higher levels of porosity produced lower IoU scores after thresholding as
the previously larger regions associated with porosity are reduced in size (Fig.
\ref{fig:ious}). Overall, the mean IoU increased for all models after the
application of a minimum pore ESD threshold.  The greatest of these increases is
seen in the models trained with the \textit{All} dataset of which scored the
highest average IoU of any other models. Rotational transforms did not prove to
have a significant impact on improving training as the resulting metrics were
often within a percent error from each other.

\begin{table}
  \begin{tabular}{ll|ll|ll}
    \hline
    \rowcolor{gray!30}
                        &                   & \multicolumn{2}{|l|}{\textit{All Pores}}  & \multicolumn{2}{l}{\textit{$\mu + \sigma$ Pores}} \\
    \rowcolor{gray!30}
    Dataset             & Data Augmentation & Average IoU    & Max IoU                  & Average IoU    & Max IoU                          \\
    \hline                                 
    \textit{Spacing}    &  Rotational       & 0.25           & 0.75                     & 0.29           & \textbf{1.0}                     \\
                        &  None             & \textbf{0.28}  & 0.77                     & 0.28           & 1.0                              \\
    \rowcolor{gray!10}                     
    \textit{Velocity}   &  Rotational       & 0.16           & 0.42                     & 0.24           & 1.0                              \\
    \rowcolor{gray!10}                     
                        &  None             & 0.17           & 0.44                     & 0.29           & 1.0                              \\
    \textit{All}        &  Rotational       & 0.22           & \textbf{0.85}            & 0.32           & 1.0                              \\
                        &  None             & 0.21           & 0.72                     & \textbf{0.32}  & \textbf{1.0}                     \\
    \hline
  \end{tabular}
  \caption{
    Pore localization prediction performance metrics for ViViT Dense model
    trained on \textit{Spacing}, \textit{Velocity}, and \textit{All} sample
    datasets for 1000 epochs on both all segmented pores and pores with
    equivalent diameters greater than 1 standard deviation above the mean. The
    adjusted threshold to consider only pores with larger equivalent diameters
    improved prediction results in build layers with nominal process parameter
    and low resulting porosity.
  }
  \label{tab:pore_localization_vivit_dense_results}
\end{table}

\subsection{Model Limitations}
The primary limitations of this work include the restricted process parameter
range in the datasets, which may hinder the model's ability to generalize to
combinations outside this range. While the dataset utilized in this study primarily contains lack of fusion porosity, pore characteristics such as size, shape, and defect origin from either keyholing, lack of fusion, or spatter can be explored in future work. Additionally,
the models are limited to raster patterns that travel parallel to the x or y
axis of the build plate. The models also require the entire sequence of thermal
images associated with layer porosity to be used as input for accurate
layer-wise predictions.  Lastly, the hardware requirements present a challenge,
as the input sequence of images demands a significant portion of the memory (over 40 gigabytes)
available on the Nvidia A6000 GPUs used for training.

\section{Conclusion and Future Work}
In this work we investigate the application of machine learning to
\textit{in-situ} thermal image process monitoring for the prediction of pore
count and pore localization. For this we utilized a CNN architecture and a
modified ViViT model with dense prediction heads for various dataset such as
\textit{Spacing}, \textit{Velocity}, and \textit{All}. For the task of pore
quantification, we have found that the \textit{Spacing} dataset provides the
greatest amount of signal within and models trained on the \textit{Velocity}
dataset produces the least amount of error. The pore localization task displayed
a similar trend with models trained on the \textit{Spacing} dataset achieving
the best overlap when evaluating \textit{All Pores}. The model trained on the
\textit{All} dataset showed better performance when evaluating on \textit{$\mu +
\sigma$ Pores}.

In both tasks, the effect of rotational transforms were minimal resulting in a
negligible difference in prediction outcomes. Our pore localization model
experienced improved performance with the application of a minimum pore ESD
threshold as it achieved higher average IoU scores, especially within areas of
the sample built with nominal processing parameters. These works show the
potential of utilizing \textit{in-situ} process monitoring techniques for faster
\textit{ex-situ} part certification and future work would aim to develop a more
robust digital twin achieving greater defect quantification and localization
precision over the entire sample. Further extension upon development of a physical
replica, known as a
\textit{Product Twin}, would lead to dynamic process parameter optimization which
can possibly fix and or prevent projected defects.

\section{Appendix}
\subsection{Input Cropping and Label Downsampling}
The input thermal images were cropped from the original 85 px \texttimes\;60 px
shape down to a 64 px \texttimes\;64 px shape to fit the desired input shape of
the network. To achieve this 8 pixels (188.8 \textmu m) were cropped from both
the top and bottom of the image and 1 pixel (23.6 \textmu m) was removed from
the left hand side of the image. To match the cropped input, the corresponding
CT label was first cropped then downscaled to align with the expected 64 px
\texttimes\;64 px input. The crops of 8 pixels and 1 pixel were converted to
their voxel equivalents and rounded to 52 voxels and 7 voxels respectively. This
reduced the CT label a size of 520 voxels \texttimes\;423 voxels down to 416
voxels \texttimes\;416 voxels (Fig.
\ref{fig:input_cropping_label_downsampling}a). The resulting cropped CT was
downscaled by a factor of 24 down to 18 voxels \texttimes\;18 voxels along the x
and y directions and 1 voxel along the z direction. Further cropping was applied
to the x and y directions by 1 voxel on each side to reduce the label down to 16
voxels \texttimes\;16 voxels\texttimes\;1 voxel (Fig.
\ref{fig:input_cropping_label_downsampling}b).

\begin{figure}[bt]
  \centering
  \includegraphics[width=\textwidth]{chapters/2_thermopore/label_cropping.png}
  \caption{
  Relative porosity locations are derived from the raw CT data which are cropped
  to capture a consistent area to match the input thermal data (Fig.
  \ref{fig:input_cropping_label_downsampling}a). The CT data is further
  downsampled factor of 24 and cropped to provide a 16 x 16 label shape (Fig.
  \ref{fig:input_cropping_label_downsampling}b).
  }
  \label{fig:input_cropping_label_downsampling}
\end{figure}

\subsection{Calculating absolute temperature estimations with data
obtained through Stratonics pyrometer}

The estimation of absolute temperature was achieved with grey-body assumption
where the emissivity remains constant at various wavelengths. Thus, with proper
calibration using a NIST-traceable tungsten lamp, the pyrometer was able to read
out a temperature estimation within 4\% accuracy for stainless steel.
\cite{mitchell_linking_2020}. This temperature is estimated with the following
formula for hybrid mode temperature estimation (Eq. \ref{eq:t_h}) where $p_2$ is
a constant held at 14388 nm-K. Wien's approximation \cite{ikeuchi_computer_2021}
of Planck's law is used to evaluate $A_\lambda$ (Eq. \ref{eq:a_lambda}) which
takes into consideration emissivity and the instrument's detection factor with
$\bar{I_1}$ representing the average intensity calculated over region
$\Omega_p$. $I_1$ and $I_2$ here represent the the radiance from images over
$\Omega_p$, for $\lambda_1 = 750 \text{nm}$ and $\lambda_2 = 900 \text{nm}$
respectively. The temperature ratio $T_R$ (Eq. \ref{eq:t_r}) is calculated with
$R$, the average of radiance from images $I_1$ and $I_2$ over $\Omega_p$, and
hardware constants for contour levels $c_1$ and $c_2$ obtained from least
squares fitting over calibration data\cite{zalameda_four-color_2016}. Contour
level defines the region $\Omega_p$ with a $\beta$ value between 0 and 1 through
means of marching squares (Eq. \ref{eq:c}). Our dataset utilizes images taken at
$\beta = 0.7$ for its most accurate approximation of the meltpool by single
contour as opposed to the multiple contours that appear at $\beta = 0.3$.

\begin{equation}
  T_H = \frac{p_2}{\lambda_2\ln\left({A_\lambda/I_2}\right)}
  \label{eq:t_h}
\end{equation}
\begin{equation}
  A_\lambda =  \bar{I}_1 e^{p_2/\lambda_1T_R}
  \label{eq:a_lambda}
\end{equation}
\begin{equation}
  T_R = \frac{1}{c_1 \ln R + c_2}
  \label{eq:t_r}
\end{equation}
\begin{equation}
  c = \beta \times \text{max}\left(I\right)
  \label{eq:c}
\end{equation}

\subsubsection{
  Sample alignment of thermal images to CT data along the X and Y
  axes
}
CT data for both samples had a shape of 1410 \texttimes\;900 \texttimes\;430
voxels with values ranging from 0 - 255 depending on data selection type of
pore, sample, or pore segmented sample. Common alignment between the
\textit{in-situ} pyrometry data and \textit{ex-situ} CT data was established by
converting the pyrometry data into voxels and applying a consistent offset.
The \textit{in-situ} pyrometry data was converted from pixels (80 px
\texttimes\;65 px) to microns (1680 \textmu m \texttimes\;1365 \textmu m) to
voxels (462.81 voxels \texttimes\;376.03 voxels). The CT data was then shifted
to align with the pyrometry data by -18 voxels \texttimes\;0 voxels
\texttimes\;63 voxels for the \textit{Spacing} dataset and -14 voxels
\texttimes\;-2 voxels \texttimes\;73 voxels for the \textit{Velocity} dataset.

\subsection{Trends of IoU scores of ViViT Dense models trained varying pore datasets}

\begin{figure}[bt]
    \centering
    \includesvg[scale=0.40]{chapters/2_thermopore/figure_9.svg}
    \caption{
        ViViT Dense model IoU trends by build layer for various datasets with and
        without rotational transforms without the application of a minimum pore
        threshold.
    }
    \label{fig:pore_localization_results}
\end{figure}

\begin{figure}[bt]
    \centering
    \includesvg[scale=0.40]{chapters/2_thermopore/figure_11.svg}
    \caption{
        ViViT Dense model IoU trends by build layer for various datasets with
        and without rotational transforms with the application of minimum
        equivalent diameter pore threshold where equivalent sphere diameter of
        pores are 1 standard deviation above the mean (\textit{$\mu + \sigma$
        Pores}).
    }
    \label{fig:pore_localization_results_thresholded}
\end{figure}
