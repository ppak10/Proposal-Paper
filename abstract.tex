\begin{abstract}
The additive manufacturing process is a complex and multi-faceted fabrication
challenge which expands in complexity at higher levels of precision.
Specifically, powder based processes such as Laser Powder Bed Fusion and
Selective Laser Sintering present challenges from overall build feasibility to
defect mitigation within the final part. Approaches to combat potential issues
have explored feedforward and feedback control of the build process utilizing
\textit{in-situ} process monitoring data, multi-physics simulation, and machine
learning models to detect and address potential defects before they arise in the
build. These approaches have shown success in limited and controlled
experimental trials and exploration of implementing these systems in a dynamic
environment is yet to be done. In this work an agentic system will developed for
the optimization of the selective laser sintering process and integrated into a
continual learning environment to further account for changing enviromental
variables.

This proposal will first explore the works achieved to develop the individual
components of this system starting from the process monitoring with
\textit{in-situ} two-color pyrometry data, to fine-tuning large language models
for domain knowledge integration, and agentic system implementation of the
intelligent automation of research tasks. These works set the foundation for the
proposed work for an agentic process optimization system within selective laser
sintering where through experimental trials and prints, the capabilities of the
system will be evaluated. The results obtained from these experiments will
further expand the dataset used to fine-tune this system into a continual
learning environment where system will better understand the dynamics of its
environment.

\end{abstract}
